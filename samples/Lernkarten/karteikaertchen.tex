\documentclass[a4paper,frontgrid]{flacards}
\usepackage{siduxflacards}

% Selbsterklärend: Beschriftungen und Farben angeben. für die Farbe kann eine Farbe aus dem Paket "color" eingetragen werden.
% Lesen Sie in der Dokumentation zu diesem Paket nach, wie Sie eigene Farben definieren und welche vordefiniert sind.

% Farbe der Balken und der Schrift auf den Balken
\renewcommand{\vorderseitenfarbebalken}{red}
\renewcommand{\rueckseitenfarbebalken}{blue}

% Achten Sie hier auf Kontrast ;)
\renewcommand{\vorderseitenfarbetext}{white}
\renewcommand{\rueckseitenfarbetext}{white}

% Überschriften, Unterschriften
\renewcommand{\ueberschriftvorne}{Seminarix Frage}
\renewcommand{\ueberschrifthinten}{Seminarix Antwort}
\renewcommand{\flfoot}{\footnotesize Frühjahr 2009}
\begin{document}

% Festlegen, wie viele Karten auf eine Seite sollen. Hier: 2 Spalten, 6 Zeilen
\pagesetup{2}{6}

% Karten direkt hier eingeben
\card{erste Frage}{erste Antwort}
\card{nächste Frage}{andere Antwort}
% mit input{} festlegen, welche Dateien eingebunden werden sollen. Auch hier gilt: Man kann Karten direkt
% hier eingeben oder sich eine kleine Sammlung in Extradateien anlegen. Der Vorteil ist, dass man die von extern
% eingebundenen Karten bei Bedarf anders weiterverwenden kann, da in den Dateien kein Overhead ist, sondern direkt
% losgelegt wird. Schauen Sie sich die Datei lernkarten.tex an.
\card{Gasgleichung: Massenpunktgas ohne anziehende Wechselwirkung.}{\begin{equation}p\, v = R\, T\end{equation}}
\card{Gasgleichung: Massenpunktgas mit anziehender Wechselwirkung.}{\begin{equation}\left[p+\frac{a}{v^2}\right]\, v = R\, T\end{equation}}
\card{Gasgleichung: Starrkugelgas ohne anziehende Wechselwirkung.}{\begin{equation}p\, (v-b) = R\, T\end{equation}}
\card{Gasgleichung: Van-der-Waals-Gas.}{\begin{equation}\left[p+\frac{a}{v^2}\right]\, (v-b) = R\, T\end{equation}}
\card{Virialgleichung realer Gase, physikalische Form.}{\begin{equation}p\, v = R\, T\, (1+B\,\rho+C\, \rho^2 + ...)\end{equation}}
\card{Virialgleichung realee Gase, technische Form.}{\begin{equation}p\, v=R\,T+B^*\, p+C^*\, p^2\end{equation}}
\card{Wie lautet die Gleichung für den Joule-Thompson-Koeffizienten $\mu$?}{\begin{equation} \mu_{JT} = \left( \frac{ \partial T }{ \partial p }\right)_H \end{equation}}
\card{Reales Gas: Zu was kann man $\left(\frac{\partial u}{\partial v}\right)_T$ umschreiben?}{\begin{equation}\left(\frac{\partial u}{\partial v}\right)_T=T\, \left(\frac{\partial p}{\partial T}\right)_v -p\end{equation}}
\card{Reales Gas: Zu was kann man $\left(\frac{\partial h}{\partial p}\right)_T$ umschreiben?}{\begin{equation}\left(\frac{\partial h}{\partial p}\right)_T=v-T\, \left(\frac{\partial v}{\partial T}\right)_p\end{equation}}
\card{Was ist die \emph{Klassische Idealkurve}?}{Diese Kurve verbindet die Punkte im \emph{pv, p}-Diagramm, auf denen $p\, v=R\, T$ für $p > 0$ gilt.}
\card{Was ist die \emph{Boyle-Kurve}?}{Die Boyle-Kurve verbindet die Punkte im \emph{pv, p}-Diagramm, für die gilt: \begin{equation}\left(\frac{\partial (p\cdot v)}{\partial p}\right)_T=0\end{equation}}
\card{Was ist charakteristisch am \emph{Boyle-Punkt}?}{In diesem Punkt ist die Steigung der Isothermen des betrachteten Gases gerade null bei $p=0$.}
\card{Wie ist das Differential $\mathrm{d}s$ definiert?}{\begin{equation}\mathrm{d}s=\frac{dh-vdp}{T}=\frac{du+pdv}{T}\end{equation}}
\card{Wie ist $c_v$ definiert?}{\begin{equation}c_v = \left(\frac{\partial u}{\partial T}\right)_v\end{equation}}
\card{Wie ist $c_p$ definiert?}{\begin{equation}c_p = \left(\frac{\partial h}{\partial T}\right)_p\end{equation}}
\card{Wie kann man das Differential $dh$ umschreiben?}{\begin{equation}dh = c_p\, dT + \left(\frac{\partial h}{\partial p}\right)_T\, dp\end{equation}}
\card{Wie ist das Differential $\mathrm{d}u$ definiert?}{\begin{equation}\mathrm{d}u=\left(\frac{\partial u}{\partial T}\right)_v\, \mathrm{d}T+\left(\frac{\partial u}{\partial v}\right)_T\, \mathrm{d}v\end{equation}, der erste Summand lässt sich wieder zu $c_v$ umschreiben.}
\card{Wenn bei einer adiabatischen Gasexpansion (etwa in einem isolierten Zylinder) die Temperatur sinkt, wie werden dann wohl die Wechselwirkungen sein?}{Die anziehenden WW werden wohl überwiegen, da die Teilchen dann beim sich-entfernen "Arbeit" aufnehmen müssen.}
\card{Ansatz für Typ \emph{... für welchen Drck ist auf der xy-Isotherme die Bedingung der klassischen Idelakurve erfüllt ...}?}{Ansatz: Zustandsgleichungen für ideales Gas gleichsetzen mit der Zustandsgleichung für das angegebene reale Gas, dabei sollte dann $R\, T$ rausfallen.}
\card{Ansatz für Typ \emph{... berechnen Sie die Boyle-Temperatur ...}?}{Ansatz: \begin{equation}\lim\limits_{p \rightarrow 0}{\left(\frac{\partial p\cdot v}{\partial p}\right)_{T_B}}=0\end{equation}}
\card{Beim Verdampfen von Wasser bleibt die Temperatur ...}{... konstant.}
\card{Wie nennt man den Stoff, der sich in der Gasphase befindet wenn die Flüssigkeit gerade verdampft ist?}{Sattdampf oder Nassdampf}
\card{Was ist die \emph{Verdampfungsenthalpie}?}{Die auf die Masse bezogene, \emph{isobare} Enthalpieerhöhung.}
\card{Wie lautet die kalorische Zustandsgleichung für $\mathrm{d}u$?}{$\mathrm{d}u = \left(\frac{\partial u}{\partial T}\right)_v \, \mathrm{d}T+ \left(\frac{\partial u}{\partial v}\right)_T \, \mathrm{d}v$ \\ \textbf{Achtung} der zweite Summand wir beim idealen Gas zu null (aus Versuchen).}
\card{Wie lautet die kalorische Zustandsgleichung für $\mathrm{d}h$?}{$\mathrm{d}u = \left(\frac{\partial h}{\partial T}\right)_p \, \mathrm{d}T+ \left(\frac{\partial h}{\partial p}\right)_T \, \mathrm{d}p$ \\ \textbf{Achtung} der zweite Summand wir beim idealen Gas zu null (aus Versuchen).}
\card{Was ist der \emph{Wassergehalt} der Luft?}{$x=\frac{m_W}{m_L}$ \\ Trockene Luft: $x=0$, gerade gesättigte Luft: $x=x_S$, Reines Wasser: $x=\infty$}
\card{Wie berechnet man den Wassergehalt der feuchten Luft aus den Partialdrücken?}{$x=0,622\, \frac{p_D}{p_L}$, kommt von der Gleichung $\frac{p_D}{p_L}=x\, \frac{M_L}{M_D}$ wenn man die molaren Massen einsetzt.}
\card{Wie ist die \emph{relative Feuchte} definiert?}{$\varphi = \frac{p_D}{p_S(t)}$ mit dem Partialdruck des Dampfes $p_D$ und dem Sättigungsdampfdruck $p_S(t)$ (aus Wasserdampftafel)}
\card{Bei bekanntem $p_D$ und $p_{ges}$ - wie errechnet sich $x$?}{$x=\frac{p_D}{p_L}\, \frac{R_L}{R_D} = \frac{p_D}{p_{ges}\,p_D}\, \frac{R_L}{R_D}$}
\card{Masse \emph{trockener Luft} aus Masse und $x$?}{$x=\frac{m_D}{m_L} \longrightarrow m_L=\frac{m}{x+1}$, hier kann man auch Massenströme einsetzen.}
\card{Wie ist ein Aspirationspsychometer aufgebaut?}{Zwei Thermometer, eines mit einem Strumpfschlauch feutgehalten, das andere trocken, beide von der Luft umströmt. Es stellen sich zwei unterschiedliche Temperaturen ein.}
\card{Idealgas, isochore Zustandsänderung: $\Delta s$?}{$\Delta s = c_v\, \mathrm{ln}\frac{T_2}{T_1} = c_v\, \mathrm{ln}\frac{p_2}{p_1}$}
\card{Idealgas, isobare Zustandsänderung: $\Delta s$?}{$\Delta s = c_p\, \mathrm{ln}\frac{T_2}{T_1} = c_p\, \mathrm{ln}\frac{v_2}{v_1}$}
\card{Idealgas, isotherme Zustandsänderung: $\Delta s$?}{$\Delta s = -R\, \mathrm{ln}\frac{p_2}{p_1} = R\, \mathrm{ln}\frac{v_2}{v_1}$}
\card{Idealgas: $\Delta h$?}{$\Delta h = c_p\, (T_2-T_1)$}
\card{Idealgas: $\Delta u$?}{$\Delta u = c_v\, (T_2-T_1)$}
\card{Idealgas: $\Delta s$?}{$\Delta s = c_p\, \mathrm{ln}\frac{T_2}{T_1}-R\, \mathrm{ln}\frac{p_2}{p_1}$}
\card{Idelagas, $v=const: \frac{p_2}{p_1}$?}{$\frac{p_2}{p_1}=\frac{T_2}{T_1}$}
\card{Idelagas, $p=const: \frac{v_2}{v_1}$?}{$\frac{v_2}{v_1}=\frac{T_2}{T_1}$}
\card{Idelagas, $T=const: \frac{p_2}{p_1}$?}{$\frac{p_2}{p_1}=\frac{v_1}{v_2}$}
\card{Idelagas, $s=const: \frac{T_2}{T_1}$?}{\begin{equation}\frac{T_2}{T_1}=\frac{p_2}{p_1}^{\frac{\kappa-1}{\kappa}}\end{equation}}
\card{Gasgleichung: Massenpunktgas ohne anziehende Wechselwirkung.}{\begin{equation}p\, v = R\, T\end{equation}}
\card{Gasgleichung: Massenpunktgas mit anziehender Wechselwirkung.}{\begin{equation}\left[p+\frac{a}{v^2}\right]\, v = R\, T\end{equation}}
\card{Gasgleichung: Starrkugelgas ohne anziehende Wechselwirkung.}{\begin{equation}p\, (v-b) = R\, T\end{equation}}
\card{Gasgleichung: Van-der-Waals-Gas.}{\begin{equation}\left[p+\frac{a}{v^2}\right]\, (v-b) = R\, T\end{equation}}
\card{Virialgleichung realer Gase, physikalische Form.}{\begin{equation}p\, v = R\, T\, (1+B\,\rho+C\, \rho^2 + ...)\end{equation}}
\card{Virialgleichung realee Gase, technische Form.}{\begin{equation}p\, v=R\,T+B^*\, p+C^*\, p^2\end{equation}}
\card{Wie lautet die Gleichung für den Joule-Thompson-Koeffizienten $\mu$?}{\begin{equation} \mu_{JT} = \left( \frac{ \partial T }{ \partial p }\right)_H \end{equation}}
\card{Reales Gas: Zu was kann man $\left(\frac{\partial u}{\partial v}\right)_T$ umschreiben?}{\begin{equation}\left(\frac{\partial u}{\partial v}\right)_T=T\, \left(\frac{\partial p}{\partial T}\right)_v -p\end{equation}}
\card{Reales Gas: Zu was kann man $\left(\frac{\partial h}{\partial p}\right)_T$ umschreiben?}{\begin{equation}\left(\frac{\partial h}{\partial p}\right)_T=v-T\, \left(\frac{\partial v}{\partial T}\right)_p\end{equation}}
\card{Was ist die \emph{Klassische Idealkurve}?}{Diese Kurve verbindet die Punkte im \emph{pv, p}-Diagramm, auf denen $p\, v=R\, T$ für $p > 0$ gilt.}
\card{Was ist die \emph{Boyle-Kurve}?}{Die Boyle-Kurve verbindet die Punkte im \emph{pv, p}-Diagramm, für die gilt: \begin{equation}\left(\frac{\partial (p\cdot v)}{\partial p}\right)_T=0\end{equation}}
\card{Was ist charakteristisch am \emph{Boyle-Punkt}?}{In diesem Punkt ist die Steigung der Isothermen des betrachteten Gases gerade null bei $p=0$.}
\card{Wie ist das Differential $\mathrm{d}s$ definiert?}{\begin{equation}\mathrm{d}s=\frac{dh-vdp}{T}=\frac{du+pdv}{T}\end{equation}}
\card{Wie ist $c_v$ definiert?}{\begin{equation}c_v = \left(\frac{\partial u}{\partial T}\right)_v\end{equation}}
\card{Wie ist $c_p$ definiert?}{\begin{equation}c_p = \left(\frac{\partial h}{\partial T}\right)_p\end{equation}}
\card{Wie kann man das Differential $dh$ umschreiben?}{\begin{equation}dh = c_p\, dT + \left(\frac{\partial h}{\partial p}\right)_T\, dp\end{equation}}
\card{Wie ist das Differential $\mathrm{d}u$ definiert?}{\begin{equation}\mathrm{d}u=\left(\frac{\partial u}{\partial T}\right)_v\, \mathrm{d}T+\left(\frac{\partial u}{\partial v}\right)_T\, \mathrm{d}v\end{equation}, der erste Summand lässt sich wieder zu $c_v$ umschreiben.}
\card{Wenn bei einer adiabatischen Gasexpansion (etwa in einem isolierten Zylinder) die Temperatur sinkt, wie werden dann wohl die Wechselwirkungen sein?}{Die anziehenden WW werden wohl überwiegen, da die Teilchen dann beim sich-entfernen "Arbeit" aufnehmen müssen.}
\card{Ansatz für Typ \emph{... für welchen Drck ist auf der xy-Isotherme die Bedingung der klassischen Idelakurve erfüllt ...}?}{Ansatz: Zustandsgleichungen für ideales Gas gleichsetzen mit der Zustandsgleichung für das angegebene reale Gas, dabei sollte dann $R\, T$ rausfallen.}
\card{Ansatz für Typ \emph{... berechnen Sie die Boyle-Temperatur ...}?}{Ansatz: \begin{equation}\lim\limits_{p \rightarrow 0}{\left(\frac{\partial p\cdot v}{\partial p}\right)_{T_B}}=0\end{equation}}
\card{Ansatz für \textit{Stahlbehälter, Nassdampf wird bis zum krit. Pkt. aufgeheizt, ...}}{Ansatz: isochor, $v=const=v_{krit}$}
\card{Dampfanteil $x$ aus den spezifischen Volumina}{\begin{equation}x_1=\frac{v_1-v'}{v''-v'}\end{equation}}
\card{Berechnung von $h$ im Nassdampfgebiet bei bekannten $h',h'', x$?}{\begin{equation}h=(1-x)\, h' + x\, h''\end{equation}}
\card{Erster Hauptsatz für durch Kontrollraum hindurchbewegtes System}{\begin{equation}q+w=\Delta u + \Delta (p\cdot v)+\frac{\Delta c^2}{2}+g \cdot \Delta z\end{equation}}
\card{Definition: Enthalpie}{$u+p\cdot v = h$}
\card{Perfektes Gas: $\Delta u$}{\begin{equation}\Delta u=c_v \cdot \Delta T\end{equation}}
\card{Perfektes Gas: $\Delta h$}{\begin{equation}\Delta h = c_p \cdot \Delta T\end{equation}}
\card{Isentropenexponent $\kappa$?}{\begin{equation}\kappa = \frac{c_p}{c_v} = \frac{c_p}{c_p-R}\end{equation}}

\end{document}