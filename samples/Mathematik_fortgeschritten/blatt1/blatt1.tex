% Dokumentenklasse festlegen
\documentclass{uebungsblatt}

% Seminarix-Paket mit einigen Befehlen laden.
% Hier jetzt das Paket laden.
\usepackage{seminarix}
% In diesen Befehlen werden teilweise Dateien eingebunden (Aufgaben, Abbildungen). Um die Einbindung so einfach wie möglich zu machen,
% werden in den folgenden beiden Zeilen Variablen definiert, die auf die Verzeichnisse zeigen, in denen Ihre Aufgaben
% und Ihre Abbildungen liegen. Vergessen Sie bitte das abschließende "/" bei Verzeichnisnamen nicht! 
% Hier im Beispiel liegen die Abbildungen und die Aufgaben im gleichen Verzeichnis ../aufgabensammlung/
\renewcommand{\afgsammlung}{../aufgabensammlung/}
\renewcommand{\bildersammlung}{../aufgabensammlung/}


% Blatttitel setzten
\blatttitel{Folgen und Reihen}



% Hier beginnt das eigentliche Aufgabenblatt.
\begin{document}

% \incafg ist ein Befehl aus dem Seminarix-Paket, er bindet eine Aufgabe ein. Die Aufgaben müssen in dem Verzeichnis liegen, welches Sie oben angegeben haben.
% Die Nummerierung erfolgt automatisch. Sehen Sie sich die eingebundenen Dateien an, wenn Sie wissen möchten, wie Sie Aufgaben erstellen können.
\incafg{folgen01}
\incafg{kurvendiskussion}
\end{document}
