\begin{aufgabe}
Die Folge $a_n$ sei wie folgt definiert:
% Die folgende Formel wird in eine eigene Zeile gedruckt. Kürzere Formeln, die im Test erscheinen sollen, werden mit $ formel $ erstellt,
% in den Teilaufgaben unten finden Sie diese Notation. Lassen Sie das Sternchen * weg, werden die Formeln durchnummeriert.
\begin{equation*}
a_{n+1} = a_n \, \left[1- \frac{3\, \pi}{\sin \pi} \right]
\end{equation*}
Geben Sie für die folgenden $a_0$ jeweils an, ob 1) die Folge konvergiert und 2) eine geschlossene Darstellung gefunden werden kann. 
% Da die Texte der folgenden Teilaufgaben sehr kurz sind, werden sie hier in mehreren Spalten (hier 2) gedruckt. 
% Löschen Sie begin{multicols} und \end{multicols}, um dieses zu unterbinden.
\begin{multicols}{2}
% Das zweite Argument "a)" gibt die Art der nummerierung an. Versuchen Sie auch z.B. A), 1), 1.
\begin{teilaufgaben}{a)}
\teilaufgabe $a_0 = 3$
\teilaufgabe $a_0 = \infty$
\teilaufgabe $a_0 = 4 \cdot \Omega$
\teilaufgabe $a_0 = 1$
\end{teilaufgaben}
\end{multicols}

\end{aufgabe}