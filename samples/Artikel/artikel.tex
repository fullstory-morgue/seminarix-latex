\documentclass{scrartcl}
\usepackage[T1]{fontenc}
\usepackage{lmodern}
\usepackage{graphicx}
\usepackage[utf8]{inputenc}
\usepackage{ngerman}
\usepackage{hyperref}
\usepackage{multirow}
\usepackage{seminarix}
\title{Jahresstatistik aptosid}
\author{Tux Thompson}
\date{\today}
\begin{document}
\maketitle
\tableofcontents

\section{Mitgliederstatistik}
Bla blubb \ldots
\subsection{Merchandising}
Auch hier interessante Fakten \ldots

\section{Vereinsarbeit}
\subsection{Präsenz auf LinuxTagen}
und hier wieder was \ldots
\subsection{Spesenabrechnungen}
\end{document}
